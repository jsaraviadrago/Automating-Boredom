\documentclass{article}

\usepackage[utf8]{inputenc}%tildes
\usepackage[spanish, mexico]{babel}%spanish
\date{}%sin fecha

\begin{document}

\title{Proyectos} \author{Juan Carlos Saravia}

\maketitle

\section*{Proyecto Innova Schools}

Innova Schools es una empresa educativa que brinda un servicio educativo en más 60 colegios a nivel Perú Méjico y Colombia. Uno de las grandes preocupaciones estratégicas era la deserción de estudiantes anual la cual estaba alrededor del 15\%. La cifra estaba muy por encima de lo esperado y dificultaba el plan de expansión. Luego de varios intentos comerciales la gerencia general pidió que el equipo de analítica busque alguna solución alterna. El equipo que lideré tenía como objetivo elaborar una solución analítica que permita, entender por qué los padres dejaban de matricular a sus hijos en las escuelas, proponer acciones y detectar los padres que tenían más riesgo de escoger otra propuesta educativa. Mi función en ese reto fue, conversar con los equipos para recolectar la información, entender a profundidad el proceso de matrícula, vincular al equipo de negocio en las propuesta de análisis y elaborar conjunto con mi equipo la solución analítica per se. El resultado que se logró obtener en 3 meses fue un modelo analítico que gracias a las recomendaciones le permitió accionar al negocio un plan de seguimiento que redujo la deserción a 8\%. Fue el primer modelo analítico de la organización y si bien se logró reducir la deserción aprendí también los límites que puede tener un modelo analítico para la obtención de resultados. 


\section*{Proyecto Alicorp}

Alicorp es una empresa enfocada en la compra de empresas de consumo masivo que tiene entre sus pilares el consumo masivo y la venta de productos en gran escala a establecimentos de comida. La empresa estaba intrduciendo la transformación digital dentro de la organización y un pilar importante es el de analítica avanzada. Sobre la base de esto, la estrategia organizacional era introducir alguna solución analítica que permitiera entrar una mayor rentabilidad al negocio. 



\end{document}